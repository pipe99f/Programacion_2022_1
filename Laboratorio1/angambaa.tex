\documentclass{article}
\title{Laboratorio 1}
\author{Andres Felipe Gamba Algarra}
\date{Marzo 21 de 2022}


\begin{document}
\maketitle

\section{}
\begin{center}
    \begin{tabular}{ |c|c|c| }
        \hline
        Caracter & ASCII & Binario\\
        \hline
        A & 65 & 01000001 \\
        n & 110 & 01101110 \\
        d & 100 & 01100100 \\
        r & 114 & 01110010 \\
        e & 101 & 01100101 \\
        s & 115 & 01110011 \\
          & 32 & 00100000 \\
        F & 70 & 01000110 \\
        e & 101 & 01100101 \\
        l & 108 & 01101100 \\
        i & 105 & 01101001 \\
        p & 112 & 01110000 \\
        e & 101 & 01100101 \\
          & 32 & 00100000 \\
        G & 71 & 01000111 \\
        a & 97 & 01100001 \\
        m & 109 & 01101101 \\
        b & 98 & 01100010 \\
        a & 97 & 01100001 \\
          & 32 & 00100000 \\
        A & 65 & 01000001 \\
        l & 108 & 01101100 \\
        g & 103 & 01100111 \\
        a & 97 & 01100001 \\
        r & 114 & 01110010 \\
        r & 114 & 01110010 \\
        a & 97 & 01100001  \\
        \hline
    \end{tabular}   
\end{center}


\section{}
\begin{center}
    \begin{tabular}{ |c|c|c| }
        \hline
        División por 2 & Cociente & Residuo \\
        \hline
        756/2 & 378 & 0 \\
        378/2 & 189 & 0 \\
        189/2 & 94 & 1 \\
        94/2 & 47 & 0 \\
        47/2 & 23 & 1 \\
        23/2 & 11 & 1 \\
        11/2 & 5 & 1 \\
        5/2 & 2 & 1 \\
        2/2 & 1 & 0 \\
        1/2 & 0 & 1 \\
        \hline
    \end{tabular}
\end{center}
\[
    756_{10} = 1011110100_2
\]
\section{}

101101110110_2 = (2^1 + 2^2 + 2^4 + 2^5 + 2^6 + 2^8 + 2^9 + 2^{11})_{10} = 2934_{10}

\section{}
\begin{center}
    \begin{tabular}{ |c|c|c|c| }
        \hline
        Decimal & Binario & Octal & Hexadecimal \\
        \hline
        0 & 0 & 0 & 0 \\
        1 & 1 & 1 & 1 \\
        2 & 10 & 2 & 2 \\
        3 & 11 & 3 & 3 \\
        4 & 100 & 4 & 4 \\
        5 & 101 & 5 & 5 \\
        6 & 110 & 6 & 6 \\
        7 & 111 & 7 & 7 \\
        8 & 1000 & 10 & 8 \\
        9 & 1001 & 11 & 9 \\
        10 & 1010 & 12 & A \\
        11 & 1011 & 13 & B \\
        12 & 1100 & 14 & C \\
        13 & 1101 & 15 & D \\
        14 & 1110 & 16 & E \\
        15 & 1111 & 17 & F \\
        16 & 10000 & 20 & 10 \\
        17 & 10001 & 21 & 11 \\
        18 & 10010 & 22 & 12 \\
        19 & 10011 & 23 & 13 \\
        20 & 10100 & 24 & 14 \\
        21 & 10101 & 25 & 15 \\
        22 & 10110 & 26 & 16 \\
        23 & 10111 & 27 & 17 \\
        24 & 11000 & 30 & 18 \\
        25 & 11001 & 31 & 19 \\
        26 & 11010 & 32 & 1A \\
        27 & 11011 & 33 & 1B \\
        28 & 11100 & 34 & 1C \\
        29 & 11101 & 35 & 1D \\
        30 & 11110 & 36 & 1E \\
        31 & 11111 & 37 & 1F \\
        32 & 100000 & 40 & 20 \\
        \hline
    \end{tabular}
\end{center}
\section{¿Cuál es el siguiente número hexadecimal al 19F?}
1A0

\end{document}
